\documentclass{ctexart}

\usepackage{listings}
\usepackage{hyperref}

\begin{document}

\setlength{\parindent}{0pt}


\title{自己实现一个简单操作系统}	
\maketitle

出于个人兴趣和学习目的,也为了在读论文做实验之余调节情绪,我决定自己写一个操作系统玩,全部自己手写,仅依赖:
\begin{itemize}
	\item 硬件设备(只支持 x86)
	\item GNU 编译器和链接器
\end{itemize}

恩,这感觉起来还是蛮酷炫的,有一种上世纪 Unix 黑客风格(自我感觉良好的错觉 :)。出于实现癖,不使用 Grub 这种成熟的 Bootloader 而是自己写(并没任何软件工程实际好处)。

\section{储备知识}

\subsection{编程语言}
\begin{itemize}
	\item The c programing language(经典如何能不看)
	\item C expert programing(可看可不看,但还是挺有意思的,当做娱乐)
	\item Professional assembly language(入门汇编速度较快,然而都是用户态程序)
	\item x86 Assembly Language Reference Manual(没全部看过,感觉比较工整全面)
\end{itemize}

\subsection{计算机体系结构}
\begin{itemize}
	\item x86 PC 汇编语言、设计与接口(很好的书,写 OS 必看;代码除 Dos 部分基本都可以在核心态跑,建议配合 qemu 虚拟机配合学习)
\end{itemize}

\subsection{操作系统}
\begin{itemize}
	\item 操作系统概念(这书,虽然比较经典易懂,算是一个基本概念的导论;然而个人觉得值得商榷啊,作者尝试尽量脱离硬件和代码讨论计算机和 OS,个人觉得比较玄)
	\item Advanced Programming in the UNIX Environment(这、姑且算吧,虽然和 OS 没有直接联系,认识下 Unix)
\end{itemize}

\subsection{数据结构与算法}
\begin{itemize}
	\item 算法导论(这书我目前也只看了一半啊,感觉写 OS 用不上这么复杂的算法)
	\item 数据结构与算法,C++ 实现(通俗易懂,算法简单;= =,然而这翔一样的 example code 是怎么回事)
\end{itemize}

\section{实用工具}

\begin{itemize}
	\item GCC \& GNU-binutils
	\item Make(参考书目 Managing-projects-with-GNU-Make)
	\item GNU-coreutils
	\item Vim
	\item Clang \& LLVM
	\item qemu
\end{itemize}

\section{项目地址}
从项目 Github 克隆仓库:
\begin{lstlisting}[frame=single]
git clone git@github.com:shibuyanorailgun/nros.git
\end{lstlisting}

\end{document}
