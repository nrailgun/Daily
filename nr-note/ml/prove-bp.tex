\documentclass{ctexart}

\usepackage{amsmath}
\usepackage{amssymb}
\usepackage{listings}
\usepackage{hyperref}

\begin{document}

\setlength{\parindent}{0pt}

\newcommand{\code}[1] {
	\lstinline{#1}
}

\author{Junyu Wu}

\title{Proving backward propagation}
\maketitle

\section{Denotes}
Let $x$ denotes network input, $y$ denotes predictions, $t$ denotes targets. The squared error is
\[
	E = \frac{1}{2} (t - y)^2.
\]
	
For neuron $j$ at layer $l$, the output $a_j^{(l)}$ is defined as:
\[
	a_j^{(l)} = \varphi(z_j^{(l)}) = \varphi(W_j^{(l)} a^{(l-1)}),
\]
where $W_j^l$ is neuron $j$ network parameters at layer $l$. A commonly used activition function
$\varphi$ is sigmoid:
\[
	\varphi(z) = \frac{1}{1 + e^{-z}},
\]
which has such a derivative:
\[
	\frac{\partial \varphi}{\partial z} = \varphi(z)(1 - \varphi(z)).
\]

\section{Derivative}
Calculating the partial derivates of the error with respect to weight $W$ using chain rule:
\[
	\frac{\partial E}{\partial W_j^{(l)}} =
		\frac {\partial E} {\partial a_j^{(l)}}
		\frac {\partial a_j^{(l)}} {\partial z_j^{(l)}}
		\frac {\partial z_j^{(l)}} {\partial W_j^{(l)}}
	,
\]
where $\frac {\partial z_j^{(l)}} {\partial W_j^{(l)}} = a^{(l-1)}$.

$\frac {\partial a_j^{(l)}} {\partial z_j^{(l)}}$ can be calculated easily from definition of $\varphi$.
If $\varphi$ is sigmoid function, we have:
\[
	\frac{\partial a_j^{(l)}}{\partial z_j^{(l)}} = \varphi(z_j^{(l)}) (1 - \varphi(z_j^{(l)})).
\]

If layer $l$ is output layer, calculating $\frac {\partial E} {\partial a_j^{(l)}}$ could be quite
straightforward:
\[
	\frac {\partial E} {\partial a_j^{(l)}} = t_j - a_j^{(l)} = t_j - y_j.
\]
Any how, if layer $l$ is an inner layer of network, finding the derivative with respect to $a_j^{(l)}$
is less obvious.

\section{Derivative with respect to inner layer parameter}
Denoting neurons at layer $l+1$, which receive input from neuron $a_j^{(l)}$, as:
\[
	\left\{ a_1^{(l+1)}, a_2^{(l+1)}, a_3^{(l+1)}, \dots, a_m^{(l+1)} \right\}.
\]
Considering $E$ as a function of neurons at layer $l+1$:
\[
	E = E( a_1^{(l+1)}, a_2^{(l+1)}, a_3^{(l+1)}, \dots, a_m^{(l+1)} ),
\]
\[
	\frac{\partial E}{\partial a_j^{(l)}}
		= \sum_{i=1}^{m}
		\frac{\partial E }{\partial a_i^{(l+1)}}
		\frac{\partial a_i^{(l+1)} }{\partial z_i^{(l+1)}}
		\frac{\partial z_i^{(l+1)} }{\partial a_j^{(l)}}
		,
\]
\[
	\frac{\partial E}{\partial a_j^{(l)}}	
		= \sum_{i=1}^{m}
		\frac{\partial E }{\partial a_i^{(l+1)}}
		\frac{\partial a_i^{(l+1)} }{\partial z_i^{(l+1)}}
		W_{i,j}^{(l+1)}
	.
\]

By now, we can calculate $\frac{\partial E}{\partial W}$ by calculate $\frac{\partial E}{\partial a}$ iteratively:
\[
	\frac{\partial E}{\partial W_j^{(l)}} =
		\begin{cases}
			(t - y) \frac {\partial a_j^{(l)}} {\partial z_j^{(l)}} a^{(l-1)} & l \textrm{ is output layer} \\
			( \sum_{i=1}^{m}
				\frac{\partial E }{\partial a_i^{(l+1)}}
				\frac{\partial a_i^{(l+1)} }{\partial z_i^{(l+1)}}
				W_{i,j}^{(l+1)}
			)
			\frac {\partial a_j^{(l)}} {\partial z_j^{(l)}} a^{(l-1)} & l \textrm{ is inner layer}
		\end{cases}
	.
\]
But there is a more efficient approach.

\section{A more efficient approach}
$\frac{\partial E }{\partial a_i^{(l+1)}} \frac{\partial a_i^{(l+1)} }{\partial z_i^{(l+1)}}$ has already calculated.
To use these intermediate results, we introduce:
\[
	\frac{\partial E}{\partial W_j^{(l)}} = \delta_j^{(l)} a^{(l-1)},
\]
where
\[
	\delta_j^{(l)} = (\sum_{i=1}^{m} \delta_i^{(l+1)} W_{i,j}^{(l+1)})
		\frac{\partial a_j^{(l)} }{\partial z_j^{(l)}}
		.
\]

\section{Vectorized approach}
Denoting paratermeters at layer $l$ as $W^{(l)}$:
\[
	\frac{\partial E}{\partial W^{(l)}} = \delta^{(l)} (a^{(l-1)})^T,
\]
where
\[
	\delta^{(l)} = (W^{(l+1)})^T * \delta^{(l+1)} .*
		\frac{\partial a^{(l)} }{\partial z^{(l)}}.
\]

\input{foot}