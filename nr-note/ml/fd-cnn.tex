\documentclass{ctexart}

\usepackage{amsmath}
\usepackage{amssymb}
\usepackage{listings}
\usepackage{hyperref}

\begin{document}

\setlength{\parindent}{0pt}

\newcommand{\code}[1] {
	\lstinline{#1}
}

\author{Junyu Wu}

\title{Deep Convolutional Network Cascasde for Facial Point Detection}
\maketitle

\section{Abstraction}
We proposed a new approach for detecting facial keypoint with 3-level carefully designed
convolutional neural networks. At each level, the outputs are fused for robust and accurate
estimations. There are 2 folds of advantages. 1st, the texture information over the entire
image is used to locate keypoints; 2nd, since the networks are trained to predict all keypoints
simutaneously, geometric constraints among keypoints are implicitly encoded.
The networks at the following two levels are trained to locally refine initial predictions and
their inputs are limited to small regions around the initial predictions.

\section{Introduction}
Facial keypoint detection problem is challenging when face images are taken with extreme poses,
lightings, and occlusions. Existing approaches can be generally devided into 2 categories:
classifying search windows, or directly predicting keypoint positions.

Many approaches update the positions of keypoints iteratively and a good initialization is
critical. In addition, many approaches face the problem that the visual features extracted are
not discriminative or not reliable enough.

To solve these problems, we proposes a cascaded regression approach for facial keypoint detection.
It effectively avoid the problem of local optimal. The remaining 2 levels of convolutional networks
refine the initial estimation of keypoints. The structure at these 2 levels are shallower, since
their task are low-level and their inputs are small local regions around the initial keypoints.

\section{Related Work}
Many significant approches used Adaboost, SVM, or random forest.

\section{Cascaded convolutional networks}
Figure 2 is an overview of our approach. Five keypoints to detect:
\begin{itemize}
	\item LE
	\item RE
	\item N
	\item LM
	\item RM
\end{itemize}

Level 1 contains 3 deep convolutional networks: F1(face), EN1(eye \& nose), and NM1(nose \& mouth).
Each predicts multiple facial points. The outputs are averaged to reduce variance. Figure 3
illustrates the strucuture of F1, which contains 4 convolutional layers followed by max pooling and
2 fully connected layers. EN1 and EM1 take the same structure but different size. Level 2 and level
3 takes local patches centered at the predicted positions and are allowed to make only small
changes. The sizes of patches and search ranges keep reducing along the cascade. The predicted
position at the last 2 levels is given by the average of 2 networks.

\subsection{Network structure selection}
Three import factors on the choice of network structure:
\begin{enumerate}
	\item Convolutional networks at the 1st level should be deep (for global non-linear features);
	\item Absolute rectification after hyperbolic tangent improve performance;
	\item Locally sharing weights of neurons on the same map inproves the performance.
\end{enumerate}

\subsection{Multi-level regression}
We find several effective ways to combine multiple convolutional networks.

\subsubsection{Multi-level regression}
The input region at the first level should be large enough to cover possible predictions, but too
large input region causes inaccurary for irrelavant areas included. The second level detection can
be done within a small region, where the disruption from other areas is reduced.

\subsubsection{Multi-network each level}
The final output of facial point can be formally expressed as:
\[
	x = \frac{x_1^{(1)} + x_2^{(1)} + \dots + x_{l_1}^{(1)}}{l_1}
		+ \sum_{i=2}^{n}
		\frac{x_1^{(i)} + x_2^{(i)} + \dots + x_{l_i}^{(i)}}{l_i}
	,
\]
for an $n$-level cascade with $l_i$ network at level $i$.

\section{Conclusion}
We proposed a effective convolutional network cascade for facial point detection.

\input{foot}
