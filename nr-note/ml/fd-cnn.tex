\documentclass{ctexart}

\usepackage{amsmath}
\usepackage{amssymb}
\usepackage{listings}
\usepackage{hyperref}

\begin{document}

\setlength{\parindent}{0pt}

\title{Deep Convolutional Network Cascasde for Facial Point Detection}
\maketitle

\section{Abstraction}

We proposed a new approach for detecting facial keypoint with 3-level carefully designed
convolutional neural networks. At each level, the outputs are fused for robust and accurate
estimations. There are 2 folds of advantages. 1st, the texture information over the entire
image is used to locate keypoints; 2nd, since the networks are trained to predict all keypoints
simutaneously, geometric constraints among keypoints are implicitly encoded.
The networks at the following two levels are trained to locally refine initial predictions and
their inputs are limited to small regions around the initial predictions.

\section{Introduction}

Facial keypoint detection problem is challenging when face images are taken with extreme poses,
lightings, and occlusions. Existing approaches can be generally devided into 2 categories:
classifying search windows, or directly predicting keypoint positions.

Many approaches update the positions of keypoints iteratively and a good initialization is
critical. In addition, many approaches face the problem that the visual features extracted are
not discriminative or not reliable enough.

To solve these problems, we proposes a cascaded regression approach for facial keypoint detection.
It effectively avoid the problem of local optimal. The remaining 2 levels of convolutional networks
refine the initial estimation of keypoints. The structure at these 2 levels are shallower, since
their task are low-level and their inputs are small local regions around the initial keypoints.

\section{Related Work}
Many significant approches used Adaboost, SVM, or random forest.

\section{Cascaded convolutional networks}

\end{document}
