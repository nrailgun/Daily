\documentclass{ctexart}

\usepackage{amsmath}
\usepackage{amssymb}
\usepackage{amsfonts}
\usepackage{graphicx}
\usepackage{bussproofs}
\usepackage{tikz}
\usetikzlibrary{automata,positioning}

\usepackage{color}
\definecolor{mygreen}{rgb}{0,0.6,0}
\definecolor{mygray}{rgb}{0.5,0.5,0.5}
\definecolor{mymauve}{rgb}{0.58,0,0.82}

\usepackage{listings}
\lstset{ %
	backgroundcolor=\color{white},   % choose the background color; you must add \usepackage{color} or \usepackage{xcolor}
	basicstyle=\footnotesize,        % the size of the fonts that are used for the code
	breakatwhitespace=false,         % sets if automatic breaks should only happen at whitespace
	breaklines=true,                 % sets automatic line breaking
	captionpos=b,                    % sets the caption-position to bottom
	commentstyle=\color{mygreen},    % comment style
	deletekeywords={...},            % if you want to delete keywords from the given language
	escapeinside={\%*}{*)},          % if you want to add LaTeX within your code
	extendedchars=true,              % lets you use non-ASCII characters; for 8-bits encodings only, does not work with UTF-8
	frame=single,	                   % adds a frame around the code
	keepspaces=true,                 % keeps spaces in text, useful for keeping indentation of code (possibly needs columns=flexible)
	keywordstyle=\color{blue},       % keyword style
	language=Octave,                 % the language of the code
	otherkeywords={*,...},           % if you want to add more keywords to the set
	numbers=left,                    % where to put the line-numbers; possible values are (none, left, right)
	numbersep=5pt,                   % how far the line-numbers are from the code
	numberstyle=\tiny\color{mygray}, % the style that is used for the line-numbers
	rulecolor=\color{black},         % if not set, the frame-color may be changed on line-breaks within not-black text (e.g. comments (green here))
	showspaces=false,                % show spaces everywhere adding particular underscores; it overrides 'showstringspaces'
	showstringspaces=false,          % underline spaces within strings only
	showtabs=false,                  % show tabs within strings adding particular underscores
	stepnumber=1,                    % the step between two line-numbers. If it's 1, each line will be numbered
	stringstyle=\color{mymauve},     % string literal style
	tabsize=2,	                   % sets default tabsize to 2 spaces
	title=\lstname                   % show the filename of files included with \lstinputlisting; also try caption instead of title
}
\lstdefinestyle{customc}{
	belowcaptionskip=1\baselineskip,
	breaklines=true,
	frame=L,
	xleftmargin=\parindent,
	language=C,
	showstringspaces=false,
	basicstyle=\footnotesize\ttfamily,
	keywordstyle=\bfseries\color{green!40!black},
	commentstyle=\itshape\color{purple!40!black},
	identifierstyle=\color{blue},
	stringstyle=\color{orange},
}
\lstdefinestyle{customasm}{
	belowcaptionskip=1\baselineskip,
	frame=L,
	xleftmargin=\parindent,
	language=[x86masm]Assembler,
	basicstyle=\footnotesize\ttfamily,
	commentstyle=\itshape\color{purple!40!black},
}
\lstset{style=customc}

\title{Advanced Software Engineering HW7}
\author{吴俊宇 15212880}

\begin{document}

\maketitle

\section{FOL Calculus}

(a)
\begin{prooftree}
	\AxiomC{*}
	\UnaryInfC{$p(x_1) \Rightarrow p(x_1), c(x_1)$}
	\UnaryInfC{$\forall x; p(x) \Rightarrow p(x_1) \lor c(x_1)$}
	\UnaryInfC{$\forall x; p(x) \Rightarrow \forall x; (p(x) \lor c(x))$}
	
	\AxiomC{*}
	\UnaryInfC{$c(x_1) \Rightarrow p(x_1), c(x_1)$}
	\UnaryInfC{$\forall x; c(x) \Rightarrow p(x_1) \lor c(x_1)$}
	\UnaryInfC{$\forall x; c(x) \Rightarrow \forall x; (p(x) \lor c(x))$}
	
	\BinaryInfC{$\forall x; p(x) \lor \forall x; c(x) \Rightarrow \forall x; (p(x) \lor c(x))$}
\end{prooftree}

(b)
\begin{prooftree}
	\AxiomC{$p(x_2) \Rightarrow p(x_1)$}
	\UnaryInfC{$\exists x; p(x) \Rightarrow p(x_1)$}
	\UnaryInfC{$\exists x; p(x) \Rightarrow \forall x; p(x)$}
\end{prooftree}

(c)
\begin{prooftree}
	\AxiomC{*}
	\UnaryInfC{$p(x_1) \Rightarrow p(x_1), q(x_1)$}
	\UnaryInfC{$p(x_1) \Rightarrow \exists x; p(x), q(x_1)$}
	\UnaryInfC{$\Rightarrow \exists x; p(x), p(x_1) \to q(x_1)$}
	
	\AxiomC{$p(x_1), q(x_2) \Rightarrow q(x_1)$}
	\UnaryInfC{$q(x_2) \Rightarrow p(x_1) \to q(x_1)$}
	\UnaryInfC{$\exists x; q(x) \Rightarrow p(x_1) \to q(x_1)$}
	
	\BinaryInfC{$\exists x; p(x) \to \exists x; q(x) \Rightarrow p(x_1) \to q(x_1)$}
	\UnaryInfC{$\exists x; p(x) \to \exists x; q(x) \Rightarrow \forall x; (p(x) \to q(x))$}
\end{prooftree}

(d)
\begin{prooftree}
\AxiomC{*}
\UnaryInfC{$f(g(a)) = g(a), g(a) = b \Rightarrow f(g(a)) = g(a)$}
\UnaryInfC{$f(g(a)) = g(a), g(a) = b \Rightarrow f(b) = g(a)$}
\UnaryInfC{$\forall x; f(g(x)) = g(x), g(a) = b \Rightarrow f(b) = g(a)$}
\end{prooftree}

(e)
\begin{prooftree}
	\AxiomC{*}
	\UnaryInfC{$p(y) \Rightarrow p(y)$}
	\UnaryInfC{$\forall x; p(x) \Rightarrow \forall y; p(y)$}
	\UnaryInfC{$\forall x; p(x), \forall x; \lnot \forall y; p(y) \Rightarrow$}
	\UnaryInfC{$\forall x; (p(x) \land \lnot \forall y; p(y)) \Rightarrow$}
	\UnaryInfC{$\forall x; \lnot (p(x) \to \forall y; p(y)) \Rightarrow$}
	\UnaryInfC{$\Rightarrow \exists x; p(x) \to (\forall y; p(y))$}
\end{prooftree}

(f)
\begin{prooftree}
	\AxiomC{*}
	\UnaryInfC{$p \Rightarrow p, q$}
	\UnaryInfC{$\Rightarrow p, p \to q$}
	\UnaryInfC{$\Rightarrow p, \forall x; p \to q$}
	\AxiomC{*}
	\UnaryInfC{$p, q \Rightarrow q$}
	\UnaryInfC{$q \Rightarrow p \to q$}
	\UnaryInfC{$\forall x; q \Rightarrow \forall x; p \to q$}
	\BinaryInfC{$p \to \forall x; q \Rightarrow \forall x; p \to q$}
\end{prooftree}

\begin{prooftree}
	\AxiomC{*}
	\UnaryInfC{$p \Rightarrow p, \forall x; q$}
	\UnaryInfC{$\Rightarrow p, p \to \forall x; q$}
	
	\AxiomC{*}
	\UnaryInfC{$p, q \Rightarrow q$}
	\UnaryInfC{$p, q \Rightarrow \forall x; q$}
	\UnaryInfC{$q \Rightarrow p \to \forall x; q$}

	\BinaryInfC{$p \to q \Rightarrow p \to \forall x; q$}
	\UnaryInfC{$\forall x; p \to q \Rightarrow p \to \forall x; q$}
\end{prooftree}

(g)
\begin{prooftree}
	\AxiomC{*}
	\UnaryInfC{$
		e_1 = e_2
		\Rightarrow
		e_1 = e_2
		$}
	\UnaryInfC{$
		e_1 = e,
		e_2 = e
		\Rightarrow
		e_1 = e_2
		$}
	\UnaryInfC{$
		m(e, e_1) = e_1,
		m(e, e_2) = e_2,
		m(e, e_1) = m(e_1, e),
		m(e, e_2) = m(e_2, e),
		m(e_1, e) = e = m(e_2, e)
		\Rightarrow
		e_1 = e_2
		$}
	\UnaryInfC{$
		m(e, e_1) = e_1,
		m(e, e_2) = e_2,
		\forall x; \forall y; m(x, y) = m(y, x),
		m(e_1, e) = e,
		m(e_2, e) = e
		\Rightarrow
		e_1 = e_2
		$}
	\UnaryInfC{$
		\forall x; m(e, x) = x,
		\forall x; \forall y; m(x, y) = m(y, x),
		m(e_1, e) = e,
		m(e_2, e) = e
		\Rightarrow
		e_1 = e_2
		$}
	\UnaryInfC{$
		\forall x; m(e, x) = x,
		\forall x; \forall y; m(x, y) = m(y, x),
		\forall x; m(e_1, x) = x \land m(e_2, x) = x
		\Rightarrow
		e_1 = e_2
		$}
	\UnaryInfC{$
		\forall x; m(e, x) = x, \forall x; \forall y; m(x, y) = m(y, x)
		\Rightarrow
		\forall x; m(e_1, x) = x \land m(e_2, x) = x \to e_1 = e_2
		$}
	\UnaryInfC{$
		\forall x; m(e, x) = x, \forall x; \forall y; m(x, y) = m(y, x)
		\Rightarrow
		\forall e_1; \forall e_2; (\forall x; m(e_1, x) = x \land m(e_2, x) = x) \to e_1 = e_2
		$}
\end{prooftree}

(h)

$c$ for car, $p$ for person, $d$ for drive, $a$ for adult, and $r$ for isRed.
\begin{prooftree}
	\AxiomC{*}
	\UnaryInfC{$
		d(p_1, c_1), r(c_1)
		\Rightarrow d(p_1, c_1), \exists p; a(p)
		$}
	\UnaryInfC{$
		d(p_1, c_1), r(c_1)
		\Rightarrow \exists c; d(p_1, c), \exists p; a(p)
		$}
	
	\AxiomC{*}
	\UnaryInfC{$
		d(p_1, c_1), r(c_1), a(p_1)
		\Rightarrow a(p_1)
		$}
	\UnaryInfC{$
		d(p_1, c_1), r(c_1), a(p_1)
		\Rightarrow \exists p; a(p)
		$}

	\BinaryInfC{$
		d(p_1, c_1), r(c_1),
		(\exists c; d(p_1, c)) \to a(p_1)
		\Rightarrow \exists p; a(p)
		$}
	\UnaryInfC{$
		d(p_1, c_1),
		\forall p; ((\exists c; d(p, c)) \to a(p)),
		r(c_1)
		\Rightarrow \exists p; a(p)
		$}
	\UnaryInfC{$
		\exists p; d(p, c_1),
		\forall p; ((\exists c; d(p, c)) \to a(p)),
		r(c_1)
		\Rightarrow \exists p; a(p)
		$}
	\UnaryInfC{$
		\forall c; \exists p; d(p, c),
		\forall p; ((\exists c; d(p, c)) \to a(p)),
		r(c_1)
		\Rightarrow \exists p; a(p)
		$}
	\UnaryInfC{$
		\forall c; \exists p; d(p, c),
		\forall p; ((\exists c; d(p, c)) \to a(p)),
		\exists c; r(c)
		\Rightarrow \exists p; a(p)
		$}
\end{prooftree}

\section{Using KeY}

All finished proofs are saved in \lstinline|*.proof| files. Formula (h) can't be proved by prover itself, and
have to be initialized manually. Check \lstinline|*.key| and \lstinline|*.proof| for details.

\section{Reserved Symbols}

Interpreting $=$ as referential identity allow
\begin{prooftree}
	\AxiomC{$t = t', [t/t']\phi \Rightarrow [t/t']\psi$}
	\UnaryInfC{$t = t', \phi \Rightarrow \psi$}
\end{prooftree}

Likewise, preserving $\ne$ works in a similar way.

\end{document}