\documentclass{ctexart}

\usepackage{amsmath}
\usepackage{amssymb}
\usepackage{amsfonts}
\usepackage{graphicx}
\usepackage[]{algorithm2e}

\bibliographystyle{plain}

\title{数据挖掘中的组合优化 —— 次模函数解决频繁项集挖掘}
\author{吴俊宇 15212880}

\newcommand{\scri}[0]{\mathcal{I}}

\begin{document}

\date{}
\maketitle
\tableofcontents
\pagebreak

\section{综述}

我主要的研究方向是计算机视觉和深度学习,偶尔涉及数据挖掘。组合优化问题是在有穷离散的解空间中,
寻找最优解 \cite{cop-wiki},属于优化问题。组合优化在许多领域有重要应用,比如人工智能、机器学习,和软件工程。

频繁项集挖掘(Frequent itemset mining)是数据挖掘中一个很流行的技术,用于探索数据之间的内在关联。
特别的,对于市场购物篮问题,每个数据点,也称为事务(Transaction),是一个数据项的集合。频繁项集挖掘的
目标是,在事务数据库中,寻找一些频繁项集,它们能表示一些数据的共性 \cite{hjw-book}。
由于数据项是有限且离散的,所以可能的组合也是有限且离散的,寻找这样的组合属于组合优化问题。

从统计角度出发,单纯的频繁项集挖掘存在一些问题,例如有些项集频繁是由于数据项本身频繁,数据项之间却
没有关联,或者数据项集过大,冗余,难以理解。Jaroslav Fowkes 提出,出现这种问题的原因是:解决了错误的问题。
用户并不关心频繁的项集,用户关心的是,“有趣”的项集,可以最有效描述事务数据库的项集\cite{this-paper}。
文章提出,挖掘“有趣”项集,也就是在一个简单但是自然的事务概率模型中最有效的项集。有趣项集可以
通过结构化 EM 方法推导 \cite{em-book}。本文旨在介绍 Jaroslav Fowkes 提出的算法,结合组合优化方法,解决数据挖掘问题。

\section{有趣频繁项集挖掘}

本节形式化“识别一个有趣频繁项集集合是否对研究事务数据库有效”的问题。
首先定义一些前置的概念和注记。数据项 $i$ 是论域 $U = \{1, 2, \dots, n\}$ 的元素,事务 $X$ 是 $U$ 的子集,
项集 $S$ 是项 $i$ 的集合。有趣项集的集合 $\mathcal I$ 是我们的目标,是 $U$ 的的幂集的子集。另外,
我们称项集 $S$ \textit{支持}事务 $X$ 如果 $S \subset X$。

\subsection{Generative Model}

我们提出一个简单的模型从有趣项集来产生事务数据库。模型的参数是一个有趣项集的集合 $\scri$ 和每个有趣项集
$S \in \scri$ 对应的伯努利概率 $\pi_S$。分别地对于数据库中每个事务 $X$,
\begin{enumerate}
	\item 对于每个 $S \in \scri$,独立决定是否在事务中包含 $S$
	$$
	z_S \sim \mathrm{Bernoulli}(\pi_S)
	$$
	\item 将事务设置为上一步选中的项集中的项的集合
	$$
	X = \bigcup \left\{ i \in S | S \in \scri, z_S = 1 \right\}
	$$
\end{enumerate}
注意模型允许一个项从多个项集中产生多次。

\subsection{Inference}

给定项集集合 $\scri$,让 $z$ 和 $\pi$ 成为每个项集 $S$ 的 $z_S$ 和 $\pi_S$ 的向量(数组),
其中项集 $S \in \scri$。
假定 $z$ 和 $\pi$ 是完全确定的,因为 $z_S$ 服从伯努利分布,显然从 Generative Model 可以得到产生事务 $X$ 的概率为
\begin{equation}
\label{3.1}
p(X, z \mid \pi) = \prod_{S \in \scri} \pi_S^{z_S} (1 - \pi_S)^{1 - z_S}
\text{ if } X = \bigcup_{z_S = 1} S \text{, otherwise } 0
\end{equation}
假定隐变量  $\pi$ 已知,那么可以是用极大似然估计 MLE 从事务 $X$ 递推得到 $z$。从式子 \ref{3.1} 显然可以
得到,极大似然的解便是极大后验分布 $p(z \mid X, \pi)$

\begin{equation}
\label{3.2}
\max_z \prod_{S \in \scri} \pi_S^{z_S} (1 - \pi_S)^{1 - z_S}
\end{equation}
$$
s.t. X = \bigcup \left\{ i \in S | S \in \scri, z_S = 1 \right\}
$$

取对数并重写式 \ref{3.2},我们得到更加标准的形式
\begin{equation}
\label{3.3}
\max_z \sum_{S \in \scri} z_S \ln \left( \frac{\pi_S}{1 - \pi_S} \right) + \ln \left(1 - \pi_S \right)
\end{equation}
$$
s.t. \sum_{S \mid i \in S} z_S \geq 1 \ \forall i \in X
$$
$$
z_S \in \{0, 1\} \ \forall S \in \scri
$$

这是一个加权集覆盖问题 \cite{weighted-set-cover},其中权重 $w_S \in \mathbb{R} $ 由下式给出
$$
w_S := \ln\left( \frac{\pi_S}{1 - \pi_S} \right)
$$
这是一个 NP 难问题,直接求解在实践中是不符合实际的。有一点非常重要,必须要注意到,加权集覆盖问题
是给定次模(\textit{Submodular})约束极大化线性函数的特例,我们可以将之公式化 \cite{young}。给出
支持事务的有趣项集的集合 $\tau$
\begin{equation}
\label{3.4}
\tau := \left\{ S \in \scri \mid S \subset X \right\}
\end{equation}
对于每个 $S \in \tau$ 一个实数权重 $w_S$,和一个非递减次模函数(\textit{submodular function})
$f: 2^\tau \to \mathbb{R}$,目标是为了找到一个最大总权重的 $C \subset \tau$,使得 $f(C) = f(\tau)$,
并且最大化 $\sum_{S \in C} w_S$。我们简单定义 $f(C)$ 为$C$ 中项的数目,即 $f(C) := | \bigcup_{S \in C} S |$。
注意到,根据构造,$f(\tau) = | X | $。

因此,我们得已使用次模函数贪心近似算法近似求解问题 \ref{3.3}。贪心算法通过反复选取项集 $S$,使得
$S$ 最大化 $w_S$ 除以尚未被选取项集覆盖的项的数目的值,构造一个 $C$。

\begin{algorithm}[H]
	\KwData{Transaction $X$, set of itemsets $\tau$ , weights $w$}
	initialize $C \leftarrow \emptyset$ \;
	\While {$f(C) \neq |X| $} {
		Choose $S \in \tau$ maximizing $\frac{w_S}{ f(C \cup {S}) - f(C) }$ \;
		$C \leftarrow C \cup {S}$ \;
	}
\caption{Greedy Weighted Set Covers}
\end{algorithm}

已经有人证明,该类问题上,贪心算法可以达到 $\ln |X| + 1$ 近似率 \cite{chivatal},并且,Feige 的不可近似性定义已经证明
这就是最优的可能近似了 \cite{feige}。

\pagebreak
\bibliography{subm-opt.bib}

\end{document}