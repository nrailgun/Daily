\documentclass{ctexart}

\usepackage{amsmath}
\usepackage{amssymb}
\usepackage{amsfonts}
\usepackage{graphicx}

\bibliographystyle{plain}

\title{数据挖掘中的组合优化 —— 次模函数解决频繁项集挖掘}
\author{吴俊宇 15212880}

\begin{document}

\maketitle
\tableofcontents
\pagebreak

\section{综述}

我主要的研究方向是计算机视觉和深度学习,偶尔涉及数据挖掘。组合优化问题是在有穷离散的解空间中,
寻找最优解 \cite{cop-wiki},属于优化问题。组合优化在许多领域有重要应用,比如人工智能、机器学习,和软件工程。

频繁项集挖掘(Frequent itemset mining)是数据挖掘中一个很流行的技术,用于探索数据之间的内在关联。
特别的,对于市场购物篮问题,每个数据点,也称为事务(Transaction),是一个数据项的集合。频繁项集挖掘的
目标是,在事务数据库中,寻找一些频繁项集,它们能表示一些数据的共性 \cite{hjw-book}。
由于数据项是有限且离散的,所以可能的组合也是有限且离散的,寻找这样的组合属于组合优化问题。

从统计角度出发,单纯的频繁项集挖掘存在一些问题,例如有些项集频繁是由于数据项本身频繁,数据项之间却
没有关联,或者数据项集过大,冗余,难以理解。Jaroslav Fowkes 提出,出现这种问题的原因是:解决了错误的问题。
用户并不关心频繁的项集,用户关心的是,“有趣”的项集,可以最有效描述事务数据库的项集\cite{this-paper}。
文章提出,挖掘“有趣”项集,也就是在一个简单但是自然的事务概率模型中最有效的项集。有趣项集可以
通过结构化 EM 方法推导 \cite{em-book}。本文旨在介绍 Jaroslav Fowkes 提出的算法,结合组合优化方法,解决数据挖掘问题。

\section{有趣频繁项集挖掘}

本节形式化“识别一个有趣频繁项集集合是否对研究事务数据库有效”的问题。$\mathcal{I}$

\pagebreak
\bibliography{subm-opt.bib}

\end{document}