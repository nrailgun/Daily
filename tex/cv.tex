% !TEX TS-program = xelatex
% !TEX encoding = UTF-8 Unicode
% !Mode:: "TeX:UTF-8"

% Template from: https://github.com/billryan/resume/tree/zh_CN
\documentclass{resume}
\usepackage{zh_CN-Adobefonts_external} % Simplified Chinese Support using external fonts (./fonts/zh_CN-Adobe/)
%\usepackage{zh_CN-Adobefonts_internal} % Simplified Chinese Support using system fonts
\usepackage{linespacing_fix} % disable extra space before next section
\usepackage{cite}

\begin{document}
\pagenumbering{gobble} % suppress displaying page number

\name{吴俊宇}

\basicInfo{
  \email{1332868391@qq.com} \textperiodcentered\ 
  \phone{13580501302} \textperiodcentered }
 
\section{\faGraduationCap\  教育背景}
\datedsubsection{\textbf{中山大学}, 广州}{2015 -- 至今}
\textit{在读硕士研究生}\ 计算机科学与技术, 预计 2018 年 6 月毕业
\datedsubsection{\textbf{中山大学}, 广州}{2011 -- 2015}
\textit{学士}\ 软件工程

\section{\faUsers\ 实习/项目经历}

\datedsubsection{\textbf{基于深度学习的人脸识别签到系统}}{2016年9月 -- 至今}
\role{C++, Caffe}{实验室项目}
\begin{onehalfspacing}
打卡签到制度存在漏签代签现象,该项目通过摄像头,基于深度学习,检测识别人脸并签到。
\begin{itemize}
  \item 对C++框架 Caffe 进行2次开发;
  \item 人脸/关键点检测模块;
  \item 人脸识别模块的网络结构设计,代码开发与模型训练,网络调优;
  \item 在2w+人干扰库和20人识别库中准确识别身份,在LFW 达到了 99.0+\% 正确率。
\end{itemize}
\end{onehalfspacing}

\datedsubsection{\textbf{阿里巴巴集团}}{2014年7月 -- 2014年9月}
\role{Ruby / Python Web 开发}{实习}
对代码托管网站 GITLAB 和 ReviewBoard 进行 2 次开发,定制拓展。
\begin{onehalfspacing}
\begin{itemize}
  \item 快速学习 Ruby / Rails / Django;
  \item 摒弃传统口令认证,支持阿里巴巴统一认证接口;
  \item 代码仓库重要操作通过阿里旺旺通知团队,避免沟通不善导致的代码冲突;
  \item 兼顾历史遗留,将 GITLAB 项目信息同步到 ReviewBoard;
  \item 修复Bug:查询项目列表信息丢失,错误的阿里旺旺信息弹出。
\end{itemize}
\end{onehalfspacing}

\datedsubsection{\textbf{基于属性的Linux内核访问控制模块}}{2015年3月 -- 2015年5月}
\role{C, Linux}{本科毕业设计}
\begin{onehalfspacing}
在内核中实现一个通过主体属性和客体属性,结合控制策略,进行访问控制的模块。
\begin{itemize}
  \item 编写内核 LSM 模块;
  \item 细粒度的访问控制;
  \item 消除了复杂场景下群组爆炸问题;
  \item 内核态代码编写难度大,代码量巨大,缺乏文档,无法使用工具库,Debug 非常困难。
\end{itemize}
\end{onehalfspacing}

\datedsubsection{\textbf{H3C 校园网客户端}}{2012年9月 -- 2012年10月}
\role{C}{个人项目}
\begin{onehalfspacing}
在中大访问网络需要通过客户端向学校网络中心拨号,但是学校提供的 Linux 拨号客户端非常难用。在大二无基础的情况,自学计算机网络基础知识和Linux原始套接字,根据H3C协议,自行实现Linux客户端,通过在链路层发送网络帧,实现协议中发起回话,验证身份,保持心跳的功能。
\end{onehalfspacing}

% Reference Test
%\datedsubsection{\textbf{Paper Title\cite{zaharia2012resilient}}}{May. 2015}
%An xxx optimized for xxx\cite{verma2015large}
%\begin{itemize}
%  \item main contribution
%\end{itemize}

\section{\faCogs\ IT 技能}
% increase linespacing [parsep=0.5ex]
\begin{itemize}[parsep=0.5ex]
  \item 编程语言:C/C++ > Python > Shell
  \item 平台:Linux
  \item Caffe:熟练使用、开发
\end{itemize}

\section{\faHeartO\ 获奖情况}
\datedline{优秀生奖学金2等奖}{2016}
\datedline{优秀生奖学金1等奖}{2015}
\datedline{ACM校赛2等奖}{2013}
\datedline{ACM校赛3等奖}{2012}
\datedline{优秀生奖学金3等奖}{2011}

\section{\faInfo\ 其他}
% increase linespacing [parsep=0.5ex]
\begin{itemize}[parsep=0.5ex]
  \item 个人博客:https://www.zybuluo.com/nrailgun/note/174078
  \item GitHub: https://github.com/nrailgun
\end{itemize}

%% Reference
%\newpage
%\bibliographystyle{IEEETran}
%\bibliography{mycite}
\end{document}
