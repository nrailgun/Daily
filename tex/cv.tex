% !TEX TS-program = xelatex
% !TEX encoding = UTF-8 Unicode
% !Mode:: "TeX:UTF-8"

% Template from: https://github.com/billryan/resume/tree/zh_CN
\documentclass{resume}
\usepackage{zh_CN-Adobefonts_external} % Simplified Chinese Support using external fonts (./fonts/zh_CN-Adobe/)
%\usepackage{zh_CN-Adobefonts_internal} % Simplified Chinese Support using system fonts
\usepackage{linespacing_fix} % disable extra space before next section
\usepackage{cite}

\begin{document}
\pagenumbering{gobble} % suppress displaying page number

\name{吴俊宇}

\basicInfo{
  \email{1332868391@qq.com} \textperiodcentered\ 
  \phone{13580501302} \textperiodcentered }
 
\section{\faGraduationCap\ 教育背景}
\datedsubsection{\textbf{中山大学}, 广州}{2015 -- 至今}
\textit{在读硕士研究生}\ 计算机科学与技术, 预计 2018 年 6 月毕业
\datedsubsection{\textbf{中山大学}, 广州}{2011 -- 2015}
\textit{学士}\ 软件工程

\section{\faUsers\ 实习/项目经历}

\datedsubsection{\textbf{基于量化的模型压缩和计算加速}}{2017年7月 -- 至今}
\role{算法工程师}{百度研究院实习}
基于乘积量化等方法,压缩文本相似度模型的计算结果,以及加速矩阵-向量乘法运算。
\begin{onehalfspacing}
\begin{itemize}
  \item 基于乘积量化,缓存并量化矩阵和子向量聚类中心的计算结果,加速矩阵-向量乘法运算。在正确率损失0.05\%的代价下,加速计算15\%。
  \item 利用AVX寄存器减少写操作,优化向量求和,进一步加速矩阵-向量乘法运算70\%。
  \item 在神经网络计算框架Lego中实现了该算法,并应用于全联接网络层和GRU网络层。
\end{itemize}
\end{onehalfspacing}

\datedsubsection{\textbf{基于深度学习的人脸识别签到系统}}{2016年9月 -- 至今}
\role{C++ 开发}{实验室项目}
\begin{onehalfspacing}
通过摄像头捕捉人脸,识别人脸,自动签到;负责开发模型。
\begin{itemize}
  \item 对C++框架 Caffe 进行2次开发;
  \item 人脸识别模块的网络结构设计,代码开发与模型训练,网络调优;
  \item 在2w+人干扰库和20人识别库中准确识别身份,在LFW 达到了 99.0+\% 正确率;
  \item 通过调整模型参数和模型结构,使得模型的显存开销降低了约40\%;
  \item 通过超分辨率预训练的方式,使得模型在16x16的低分辨率人脸获得了4\%的正确率提升。
\end{itemize}
\end{onehalfspacing}

\datedsubsection{\textbf{Python / Ruby Web 开发}}{2014年7月 -- 2014年9月}
\role{开发工程师}{阿里巴巴集团实习}
对代码托管网站 GITLAB 和 ReviewBoard 进行 2 次开发,定制拓展。
\begin{onehalfspacing}
\begin{itemize}
  \item 快速学习 Ruby / Rails / Django;
  \item 摒弃传统口令认证,支持阿里巴巴统一认证接口;
  \item 代码仓库重要操作通过阿里旺旺通知团队,避免沟通不善导致的代码冲突;
  \item 兼顾历史遗留,将 GITLAB 项目信息同步到 ReviewBoard;
  \item 修复Bug:查询项目列表信息丢失,错误的阿里旺旺信息弹出。
\end{itemize}
\end{onehalfspacing}

% Reference Test
%\datedsubsection{\textbf{Paper Title\cite{zaharia2012resilient}}}{May. 2015}
%An xxx optimized for xxx\cite{verma2015large}
%\begin{itemize}
%  \item main contribution
%\end{itemize}

\section{\faCogs\ IT 技能}
% increase linespacing [parsep=0.5ex]
\begin{itemize}[parsep=0.5ex]
  \item 编程语言:C/C++ / Python / Shell
  \item 平台:Linux
\end{itemize}

\section{\faHeartO\ 获奖情况}
\datedline{优秀生奖学金2等奖}{2016}
\datedline{优秀生奖学金1等奖}{2015}
\datedline{ACM校赛2等奖}{2013}
\datedline{ACM校赛3等奖}{2012}
\datedline{优秀生奖学金3等奖}{2011}

\section{\faInfo\ 其他}
% increase linespacing [parsep=0.5ex]
\begin{itemize}[parsep=0.5ex]
  \item 个人博客:https://www.zybuluo.com/nrailgun/note/174078
  \item GitHub: https://github.com/nrailgun
\end{itemize}

%% Reference
%\newpage
%\bibliographystyle{IEEETran}
%\bibliography{mycite}
\end{document}
