\documentclass{ctexart}

\usepackage{amsmath}
\usepackage{amssymb}
\usepackage{amsfonts}
\usepackage{graphicx}

\bibliographystyle{plain}

\title{计算机视觉中的人脸关键点}
\author{吴俊宇 15212880}

\begin{document}

\maketitle

人脸关键点检测(\textbf{Facial landmark localization})在许多相关的人脸相关问题(包括生物特征识别、情绪理解和人脸识别)中扮演了重要角
色,而这些问题都是是计算机视觉中的重要任务。人脸关键点检测问题事实上是非常具有挑战性的,因为这个问题涉及了空间尺度、人像姿势、表
情、人脸遮挡、光照和外貌差异。这篇文章主要讨论最近几年较为著名的人脸关键点检测技术。

关键点(landmark)这个翻译可能不尽准确,通常地标是指一些自然或人造的特征,这些特征与环境分离,清晰可见,指引方向。人脸关键点是指人脸上的若干关键位置,这些关键位置在后续的人脸相关识别任务中(人脸识别、人脸追踪,表情捕捉),起到了重要的作用。人脸关键点是优秀的特征,在许多问题中是重要的踏脚石。人脸关键点被分为两个大类,基准关键点和辅助关键点。嘴角、鼻尖这些部位可以较为容易的被低级图像特征(例如 SIFT 或 HoG)找到。这些比较容易找到的关键点通常称为基准关键点。基准关键点在人脸识别等任务中,起到了重要作用。辅助关键点的寻找被基准关键点所指导,通常是脸颊,眉毛等等。这些辅助关键点在表情理解中起到重要作用。

\begin{figure}
\centering
\includegraphics[width=0.7\linewidth]{pics/fp-example}
\caption[]{关键点检测例子}
\label{fig:fp-example}
\end{figure}

\section{人脸关键点技术的应用}

本章节列举人脸关键点通常被使用的应用场景。通常人脸关键点技术应用于人脸识别,表情分析,3D 人脸重建,人脸追踪,监控系统等等应用之中\cite{survey}。

\begin{description}
	\item[人脸识别] 人脸识别根据关键点定位(五官等)区域,提取有用特征。定位的坐标点也给给出了许多位置信息,比如角度和脸部五官之间的距离。
	\item[表情理解] 表情是没有声音和文字的语言,坐标点的空间格局提供了分析面部表情的工具。
	\item[人脸跟踪] 大部分人脸跟踪算法基于跟踪关键点序列。
\end{description}

\section{定位人脸关键点面临的挑战}

虽然人脸关键点识别在概念上非常简单,不过实际的计算机视觉问题中,却还存在许多挑战。首先对于类似于监控系统的软件,必须在嵌入式设备的有限运算力下,运行于实时;其次,实际应用中,算法健壮性尤为重要,必须对于形变、光照、姿势等等影响因素保持健壮。

在定位人脸关键点问题中主要的 3 个挑战:

\begin{description}
	\item[多变性] 关键点的外观受到了遮挡物(比如眼镜),姿势,光照,甚至是摄像头分辨率的影响,还受到人脸固有的个体差异的影响。
	有时候,手臂或者头发可能遮挡关键点的观察,光照和面部表情都让工作难上加难。目前还没有可以在计算资源有限的嵌入式设备上健壮的、
	有效的、实时的算法。
	\item[泛用性] 许多算法受到数据集的影响,一定程度上是 over-fit 的,在实际运用中,往往存在健壮性问题。
	\item[缺乏可靠的公开数据集] 虽然目前有很多公开的数据集,人脸关键点领域还缺乏公认的、低错误率的数据集(一些数据
	集精确度一般)。这严重妨碍了实际研究的展开。
\end{description}

\section{人脸关键点定位技术}

\subsection{Component based Deformable Model}

Yuchi Huang 提出 Component based Deformable Model(CDM),用于寻找人脸轮廓(Face alignment),通过统计学方法计算
局部与全局形状特征 \cite{cdm}。对 Shape components 使用独立的高斯模型,保留了更多的局部形态特征。更好地表示了 shape component 之间
的非线性相互关系。

\subsection{Deep Convolutional Network Cascade}

汤晓鸥等人提出一个三级连深度卷积神经网络模型(见图 \ref{fig:cascaded_cnn}),每个级联存在多个网络,意在希望多个网络可以给出更加健壮的输出 \cite{cascaded_cnn}。

\begin{figure}
\centering
\includegraphics[width=0.9\linewidth]{pics/cascaded_cnn}
\caption[]{Three-level cascaded convolutional networks}
\label{fig:cascaded_cnn}
\end{figure}

在第一级联中,卷积神经网络在整个面部区域提取了高阶特征。在整个面部区域,许多的上下文信息都被用来参与寻找关键点。另外,
由于 5 个关键点同时被网络所预测,可以一定程度上保持面部关键点之间的空间约束。因此,网络得以避免二义性和数据损坏导致的
局部最优。在 2 级和 3 级级联中,网络逐步微调输出结果,以期达到更好的预测结果准确性。

这个网络结构使用于关键点不多的情况,随着关键点增多,网络收敛会越来越困难,也越来越不准确。

\subsection{Coarse-to-fine Convolutional Network Cascade}

3 级联卷积神经网络表现出了良好的准确率,4 级联卷积神经网络表现更加惊人 \cite{cascaded4_cnn},将问题分解为 Coarse-to-fine,由粗到细的过程。网络的每一个级联调整上一个级联面部关键点的预测结果,预测更加精准的空间约束。

网络第一级联预测得到 bounding box 和轮廓点;第二级联预测一个初始的位置预测;第三级联和最后的级联提取预测位置的 Patch(包括
旋转变换),进行更加精准的调整。

这个方法流程较为复杂,但是适用于关键点更多的场合。

\subsection{Supervised Descent Mothod}

\cite{sdm}

\section{结语}

本文主要讨论了关键点识别问题及其遇到的困难,和几种较为著名的方法,重点介绍了 Cascaded CNN 和 SDM 方法。人脸关键点
的工作是计算机视觉中非常重要的问题,因此研究这个问题很有意义。这篇文章大部分来自我读的 Paper 和实验的笔记,今天写出来
分享,谢谢老师。

\pagebreak
\bibliography{fp-survey.bib}

\end{document}