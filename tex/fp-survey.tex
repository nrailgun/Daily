\documentclass{ctexart}

\usepackage{amsmath}
\usepackage{amssymb}
\usepackage{amsfonts}
\usepackage{graphicx}

\title{计算机视觉中的人脸关键点}
\author{吴俊宇 15212880}

\begin{document}

\maketitle

人脸关键点检测(Facial landmark localization)在许多相关的人
脸相关问题(包括生物特征识别、情绪理解和人脸识别)中扮演了重要角
色,而这些问题都是是计算机视觉中的重要任务。人脸关键点检测问题事
实上是非常具有挑战性的,因为这个问题涉及了空间尺度、人像姿势、表
情、人脸遮挡、光照和外貌差异。这篇文章主要讨论最经典和最近几年较为
著名的人脸关键点检测技术。

\section{综述}

关键点(landmark)这个翻译可能不尽准确,通常地标是指一些自然
或人造的特征,这些特征与环境分离,清晰可见,指引反向。人脸关键点
是指人脸上的若干关键位置,这些关键位置在后续的人脸相关识别任务中
(人脸识别、人脸追踪,表情捕捉),起到了重要的作用。人脸关键点是优秀
的特征,在许多问题中是重要的踏脚石。
人脸关键点被分为两个大类,基准关键点和辅助关键点。嘴角、鼻尖这
些部位可以较为容易的被低级图像特征(例如 SIFT 或 HoG)找到。这些
比较容易找到的关键点通常称为基准关键点。基准关键点在人脸识别等任
务中,起到了重要作用。辅助关键点的寻找被基准关键点所指导,通常是脸
颊,眉毛等等。这些辅助关键点在表情理解中起到重要作用。

\section{应用}

\end{document}