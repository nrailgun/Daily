\documentclass{ctexart}

\usepackage{amsmath}
\usepackage{amssymb}
\usepackage{amsfonts}
\usepackage{mathabx}
\usepackage{listings}
\usepackage{bussproofs}
\usepackage{tikz}
\usetikzlibrary{automata,positioning}

\usepackage{color}

\definecolor{mygreen}{rgb}{0,0.6,0}
\definecolor{mygray}{rgb}{0.5,0.5,0.5}
\definecolor{mymauve}{rgb}{0.58,0,0.82}

\lstset{ %
	backgroundcolor=\color{white},   % choose the background color; you must add \usepackage{color} or \usepackage{xcolor}
	basicstyle=\footnotesize,        % the size of the fonts that are used for the code
	breakatwhitespace=false,         % sets if automatic breaks should only happen at whitespace
	breaklines=true,                 % sets automatic line breaking
	captionpos=b,                    % sets the caption-position to bottom
	commentstyle=\color{mygreen},    % comment style
	deletekeywords={...},            % if you want to delete keywords from the given language
	escapeinside={\%*}{*)},          % if you want to add LaTeX within your code
	extendedchars=true,              % lets you use non-ASCII characters; for 8-bits encodings only, does not work with UTF-8
	frame=single,	                   % adds a frame around the code
	keepspaces=true,                 % keeps spaces in text, useful for keeping indentation of code (possibly needs columns=flexible)
	keywordstyle=\color{blue},       % keyword style
	language=Octave,                 % the language of the code
	otherkeywords={*,...},           % if you want to add more keywords to the set
	numbers=left,                    % where to put the line-numbers; possible values are (none, left, right)
	numbersep=5pt,                   % how far the line-numbers are from the code
	numberstyle=\tiny\color{mygray}, % the style that is used for the line-numbers
	rulecolor=\color{black},         % if not set, the frame-color may be changed on line-breaks within not-black text (e.g. comments (green here))
	showspaces=false,                % show spaces everywhere adding particular underscores; it overrides 'showstringspaces'
	showstringspaces=false,          % underline spaces within strings only
	showtabs=false,                  % show tabs within strings adding particular underscores
	stepnumber=1,                    % the step between two line-numbers. If it's 1, each line will be numbered
	stringstyle=\color{mymauve},     % string literal style
	tabsize=2,	                   % sets default tabsize to 2 spaces
	title=\lstname                   % show the filename of files included with \lstinputlisting; also try caption instead of title
}

\lstdefinestyle{customc}{
	belowcaptionskip=1\baselineskip,
	breaklines=true,
	frame=L,
	xleftmargin=\parindent,
	language=C,
	showstringspaces=false,
	basicstyle=\footnotesize\ttfamily,
	keywordstyle=\bfseries\color{green!40!black},
	commentstyle=\itshape\color{purple!40!black},
	identifierstyle=\color{blue},
	stringstyle=\color{orange},
}

\lstdefinestyle{customasm}{
	belowcaptionskip=1\baselineskip,
	frame=L,
	xleftmargin=\parindent,
	language=[x86masm]Assembler,
	basicstyle=\footnotesize\ttfamily,
	commentstyle=\itshape\color{purple!40!black},
}

\lstset{escapechar=@,style=customc}

\begin{document}

\section*{}
\textbf{15.5} Suppose $A_1, A_2, A_3, \dots$ are sentences such that no $A_n$ is provable from the
conjunction of the $A_m$ for $m < n$. Let $T$ be the theory consisting of all sentences
provable from the $A_i$ . Show that $T$ is not finitely axiomatizable, or in other
words, that there are not some other, finitely many, sentences $B_1 , B_2 , \dots , B_m$
such that $T$ is the set of consequences of the $B_j$ .

\subsubsection*{Solution:}

\section*{}
\textbf{15.7} A sentence $D$ is finitely valid if every finite interpretation is a model of $D$.
Outline an argument \textit{assuming Church’s thesis} for the conclusion that the
set of sentences that are not finitely valid is semirecursive. (It follows from
Trakhtenbrot’s theorem, as in the problems at the end of chapter 11, that the
set of such sentences is not recursive.)

\subsubsection*{Solution:}

\section*{}
\textbf{15.10} Let $T$ be an axiomatizable theory in the language of arithmetic. Let $f$ be a
one-place total or partial function $f$ of natural numbers, and suppose there is
a formula $\phi(x, y)$ such that for any $a$ and $b$, $\phi(a, b)$ is a theorem of $T$ if and
only if $f (a) = b$. Show that $f$ is a recursive total or partial function.

\subsubsection*{Solution:}

\section*{}
\textbf{16.1} Show that the class of arithmetical relations is closed under substitution of
recursive total functions. In other words, if $P$ is an arithmetical set and $f$ a
recursive total function, and if $Q(x) \leftrightarrow P( f (x))$, then $Q$ is an arithmetical set,
and similarly for $n$-place relations and functions.

\subsubsection*{Solution:}

\section*{}
\textbf{16.12} Continuing the preceding problems, prove the associative and distributive and
commutative laws for multiplication:
\begin{itemize}
	\item[(e)] $x \cdot (y + z) = x \cdot y + x \cdot z$
	\item[(f)] $x \cdot (y \cdot z) = (x \cdot y) \cdot z$
	\item[(g)] $x \cdot y = y \cdot x$
\end{itemize}

\subsubsection*{Solution:}

\section*{}
\textbf{16.14} Consider a nonstandard interpretation of the language ${0, ', <}$ in which the
domain is the set of natural numbers, but the denotation of $<$ is taken to be the
relation $<_1$ of Problem 16.13(a). Show that by giving suitable denotations to
$0$ and $'$, axioms (Q1)–(Q2) and (Q7)–(Q10) of $Q$ can be made true, while the
sentence $\forall x\; (x = 0 \lor \exists y\; x = y')$ is made false.

\subsubsection*{Solution:}

\end{document}
