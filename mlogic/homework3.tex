\documentclass{ctexart}

\usepackage{amsmath}
\usepackage{amssymb}
\usepackage{amsfonts}
\usepackage{mathabx}
\usepackage{listings}
\usepackage{tikz}
\usetikzlibrary{automata,positioning}

\usepackage{color}

\definecolor{mygreen}{rgb}{0,0.6,0}
\definecolor{mygray}{rgb}{0.5,0.5,0.5}
\definecolor{mymauve}{rgb}{0.58,0,0.82}

\lstset{ %
	backgroundcolor=\color{white},   % choose the background color; you must add \usepackage{color} or \usepackage{xcolor}
	basicstyle=\footnotesize,        % the size of the fonts that are used for the code
	breakatwhitespace=false,         % sets if automatic breaks should only happen at whitespace
	breaklines=true,                 % sets automatic line breaking
	captionpos=b,                    % sets the caption-position to bottom
	commentstyle=\color{mygreen},    % comment style
	deletekeywords={...},            % if you want to delete keywords from the given language
	escapeinside={\%*}{*)},          % if you want to add LaTeX within your code
	extendedchars=true,              % lets you use non-ASCII characters; for 8-bits encodings only, does not work with UTF-8
	frame=single,	                   % adds a frame around the code
	keepspaces=true,                 % keeps spaces in text, useful for keeping indentation of code (possibly needs columns=flexible)
	keywordstyle=\color{blue},       % keyword style
	language=Octave,                 % the language of the code
	otherkeywords={*,...},           % if you want to add more keywords to the set
	numbers=left,                    % where to put the line-numbers; possible values are (none, left, right)
	numbersep=5pt,                   % how far the line-numbers are from the code
	numberstyle=\tiny\color{mygray}, % the style that is used for the line-numbers
	rulecolor=\color{black},         % if not set, the frame-color may be changed on line-breaks within not-black text (e.g. comments (green here))
	showspaces=false,                % show spaces everywhere adding particular underscores; it overrides 'showstringspaces'
	showstringspaces=false,          % underline spaces within strings only
	showtabs=false,                  % show tabs within strings adding particular underscores
	stepnumber=1,                    % the step between two line-numbers. If it's 1, each line will be numbered
	stringstyle=\color{mymauve},     % string literal style
	tabsize=2,	                   % sets default tabsize to 2 spaces
	title=\lstname                   % show the filename of files included with \lstinputlisting; also try caption instead of title
}

\lstdefinestyle{customc}{
	belowcaptionskip=1\baselineskip,
	breaklines=true,
	frame=L,
	xleftmargin=\parindent,
	language=C,
	showstringspaces=false,
	basicstyle=\footnotesize\ttfamily,
	keywordstyle=\bfseries\color{green!40!black},
	commentstyle=\itshape\color{purple!40!black},
	identifierstyle=\color{blue},
	stringstyle=\color{orange},
}

\lstdefinestyle{customasm}{
	belowcaptionskip=1\baselineskip,
	frame=L,
	xleftmargin=\parindent,
	language=[x86masm]Assembler,
	basicstyle=\footnotesize\ttfamily,
	commentstyle=\itshape\color{purple!40!black},
}

\lstset{escapechar=@,style=customc}

\newcommand{\unsim}{\mathord{\sim}}

\begin{document}

\section*{}
\textbf{8.3} The (positive) semicharacteristic function of a set $A$ is the function $c$ such
that $c(a) = 1$ if $a$ is in $A$, and $c(a)$ is undefined otherwise. Show that a set $A$ is
recursively enumerable if and only if its semicharacteristic function is recursive.

\textbf{Solution}:

If $c$ is recursive, and $A$ is the domain of recursive function $c$, then $A$ is semirecursive.

If $A$ is semirecursive, then
$$
c(a) = b \leftrightarrow a \in A \And 1 = b
$$
$a \in A \And 1 = b$ is semirecursive (since both $a \in A$ and $\And 1 = b$ are). Function $c(a)$
of semirecursive graph relation is recursive.

\section*{}
\textbf{8.10} Give an example of a recursive partial function f such that f
cannot be extended to a recursive total function, or in other words, such that there is no
recursive total function g such that g(x) = f (x) for all x in the domain of f.

\textbf{Solution}:

Let set $D_f$ denotes domain of partial function $f$. If $D_f$ is recursive
then $g$ can extend $f$:
$$
g(x) = \begin{cases}
	f(x) & x \in D_f \\
	0 & x \not\in D_f
\end{cases}
$$

If $D_f$ is not recursive, then $g$ cannot be recursive. Since $f$ is recursive,
obviously $D_f$ is semirecursive, but it still can be non-recursive. Construct
$$
f(x) = 1, x \in D_f
$$
where $D_f(x) \leftrightarrow F(x, x) = 0$, $F$ is universal function. $F(x, x) = 0$ is not a
recursive relation, thus $g(x)$ cannot be recursive.

\section*{}
\textbf{9.5} Prove that every formula $F$ has a formation sequence in which the only formulas
that appear are subformulas of $F$, and the number of formulas that appear is no greater than
the number of symbols in $F$.

\textbf{Solution}:

A formula in formation sequence is constructed with earlier formulas by negation, connection, and
quantification, and it and its subformulas become subformulas of later formula. Finally some
formulas are used to construct last formula $F$, and every formula become subformula of $F$.

To show the number of formulas that appear is no greater than the number of symbols in $F$:
\begin{itemize}
	\item Base step: If $G$ in sequence is atomic, then the only formula in formation sequence is $F$ itself,
	the number of formula is $1$, not more than than the number of symbols in $F$, which is not less
	than $1$.
	
	\item Induction step, negation case: $\unsim G$ add one symbol to formula, and one formula to formation
	sequence. With the hypothesis holds for subformula $G$, the hypothesis holds still.
	
	\item Induction step, junction case: $(G \And H)$, $(G \lor H)$ add one symbol to formula, and one formula to formation
	sequence. With the hypothesis holds for subformula $G$ and $H$, the hypothesis holds still.
	
	\item Induction step, quantification case: $\forall x\; G$, $\exists x\; G$ add one symbol to formula, and one formula to formation
	sequence. With the hypothesis holds for subformula $G$, the hypothesis holds still.
\end{itemize}

\section*{}
\textbf{10.6} Show that:
\begin{itemize}
\item[(a)] $\{C_1 , \dots , C_m \}$ is unsatisfiable if and only if $\unsim C_1 \lor \dots \lor \unsim C_m$ is valid.
\item[(b)] $D$ is a consequence of $\{C_1 , \dots , C_m \}$ if and only if $\unsim C_1 \lor \dots \lor \unsim C_m \lor D$
is valid.
\item[(c)] $D$ is a consequence of $\{C_1 , \dots , C_m \}$ if and only if $\{C_1 , \dots , C_m, \unsim D\}$ is
unsatisfiable.
\item[(d)] $D$ is valid if and only if $\unsim D$ is unsatisfiable.
\end{itemize}

\textbf{Solution}:

\begin{itemize}

\item[(a)] If there is no interpretation that makes every formula in $\{C_1 , \dots , C_m \}$ true, then at least one
of $\{C_1 , \dots , C_m \}$ must be false, that's, at least one of $\{ \unsim C_1, \dots, \unsim C_m \}$ must
be true. Thus, $\unsim C_1 \lor \dots \lor \unsim C_m$ must be true.
If $\unsim C_1 \lor \dots \lor \unsim C_m$ is valid, then at least one of $\{ \unsim C_1, \dots, \unsim C_m \}$ must
be true, that's, at least one of $\{C_1 , \dots , C_m \}$ must be false. Thus, $\{C_1 , \dots , C_m \}$ is unsatisfiable.

\item[(b)]
We have $\unsim C_1 \lor \dots \lor \unsim C_m \lor D = \unsim (C_1 \And \dots \And C_m) \lor D
= (C_1 \And \dots \And C_m) \to D$.
If $(C_1 \And \dots \And C_m) \to D$ is valid, then no interpretation makes $(C_1 \And \dots \And C_m)$ true
and $D$ false, thus $D$ is a consequence of $\{C_1 , \dots , C_m \}$. Vice versa, if
$(C_1 \And \dots \And C_m) \to D$ is not valid, then some interpretation makes $(C_1 \And \dots \And C_m)$ true
and $D$ false, thus $D$ is not a consequence of $\{C_1 , \dots , C_m \}$.

\item[(c)] We have $\{C_1 , \dots , C_m, \unsim D\} = C_1 \And \dots \And C_m \And \unsim D
= \unsim (\unsim (C_1 \And \dots \And C_m) \lor D)
= \unsim ((C_1 \And \dots \And C_m) \to D)
$. If $\unsim ((C_1 \And \dots \And C_m) \to D)$ is unsatisfiable, then $((C_1 \And \dots \And C_m) \to D)$
is valid, that's, $D$ is a consequence of $\{C_1 , \dots , C_m \}$.
If $\unsim ((C_1 \And \dots \And C_m) \to D)$ is not unsatisfiable, then $((C_1 \And \dots \And C_m) \to D)$
is not valid, that's, $D$ is not a consequence of $\{C_1 , \dots , C_m \}$.

\item[(d)] If $\unsim D$ is unsatisfiable, then no interpretation makes $D$ false, thus $D$ is valid.
If $\unsim D$ is not unsatisfiable, then some interpretation makes $D$ false, thus $D$ is not valid.

\end{itemize}

\section*{}
\textbf{10.8}
Show that:
\begin{itemize}
\item[(a)] $(B \And C)$ implies $B$ and implies $C$.
\item[ (b)] $\unsim B$ implies $\unsim(B \And C)$, and $\unsim C$ implies $\unsim (B \And C)$.
\item [(c)] $\forall x\; B(x)$ implies $B(t)$.
\item [(d)] $\unsim B(t)$ implies $\unsim \forall x B(x)$.
\end{itemize}

\textbf{Solution}:
\begin{itemize}

\item[(a)] If $B$ is false, then $(B \And C)$ is false. The same for $C$.

\item[(b)] If $\unsim B$, that's, $B$ is false, then $(B \And C)$ is false, and $\unsim(B \And C)$ is true.

\item[(c)] If every element $m$ of the domain satisfy $B(x)$, then $t$ satisfy $B(x)$, whatever its denotion
is.

\item[(d)] Element $m$ denoted by $t$ doesn't satisfy $B(x)$, that's, not every element in domain satisfy $B(x)$.

\end{itemize}

\section*{}
\textbf{10.14}
Show that the following pairs are equivalent:
\begin{itemize}
\item [(a)] $\forall x F(x) \And \forall y G(y)$ and $\forall u(F(u) \And G(u))$.
\item [(b)] $\forall x F(x) \lor \forall y G(y)$ and $\forall u\forall v(F(u) \lor G(v))$.
\item [(c)] $\exists x F(x) \And \exists y G(y)$ and $\exists u\exists v(F(u) \And G(v))$.
\item [(d)] $\exists x F(x) \lor \exists y G(y)$ and $\exists u(F(u) \lor G(u))$.
\end{itemize}

\textbf{Solution}:
\begin{itemize}

\item[(a)] If for all element in domain $F$ holds, and for all element in domain $G$ holds, then for all element $u$
in domain, $F(u)$ holds and $G(u)$ holds. If $F$ doesn't hold for all element,
then for all element, $F$ and $G$ can't hold. The same if $G$ doesn't hold for all element.

\item[(b)] If $\forall x\; F(x)$ holds, for whatever $v$, $F(u) \lor G(v)$ holds. The same to $\forall y\; G(y)$.
If neither $\forall x\; F(x)$ nor $\forall y\; G(y)$ holds, there must be $u$ and $v$ such that
$\unsim F(u)$ and $\unsim G(v)$, that's, $\unsim (F(u) \lor G(v))$, which fail $\forall u\forall v(F(u) \lor G(v))$.

\item[(c)] If there is $x$ such that $F$ holds and $y$ such that $G$ holds, then $F(x) \And G(y)$ holds, which means
$\exists u\exists v(F(u) \And G(v))$ holds. If there is no $x$ such $F$ holds, then $F(x) \And G(y)$ can't hold
since $F(x)$ never holds, that's, $\exists u\exists v(F(u) \And G(v))$ doesn't hold. The same if $G(y)$ never holds.

\item[(d)] If $F(x)$ holds for some $x$, $F(x) \lor G(y)$ holds, that's $\exists u(F(u) \lor G(u))$. The same if $G(y)$ holds
for some $y$. If neither $\exists x\; F(x)$ nor $\exists y\; G(y)$ holds, then there is no element make $F$ or $G$ true,
that's, $F(u) \lor G(u)$ never holds. Thus, $\exists u(F(u) \lor G(u))$ is false.

\end{itemize}

\end{document}
