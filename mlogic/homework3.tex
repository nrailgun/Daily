\documentclass{ctexart}

\usepackage{amsmath}
\usepackage{amssymb}
\usepackage{amsfonts}
\usepackage{mathabx}
\usepackage{listings}
\usepackage{tikz}
\usetikzlibrary{automata,positioning}

\usepackage{color}

\definecolor{mygreen}{rgb}{0,0.6,0}
\definecolor{mygray}{rgb}{0.5,0.5,0.5}
\definecolor{mymauve}{rgb}{0.58,0,0.82}

\lstset{ %
	backgroundcolor=\color{white},   % choose the background color; you must add \usepackage{color} or \usepackage{xcolor}
	basicstyle=\footnotesize,        % the size of the fonts that are used for the code
	breakatwhitespace=false,         % sets if automatic breaks should only happen at whitespace
	breaklines=true,                 % sets automatic line breaking
	captionpos=b,                    % sets the caption-position to bottom
	commentstyle=\color{mygreen},    % comment style
	deletekeywords={...},            % if you want to delete keywords from the given language
	escapeinside={\%*}{*)},          % if you want to add LaTeX within your code
	extendedchars=true,              % lets you use non-ASCII characters; for 8-bits encodings only, does not work with UTF-8
	frame=single,	                   % adds a frame around the code
	keepspaces=true,                 % keeps spaces in text, useful for keeping indentation of code (possibly needs columns=flexible)
	keywordstyle=\color{blue},       % keyword style
	language=Octave,                 % the language of the code
	otherkeywords={*,...},           % if you want to add more keywords to the set
	numbers=left,                    % where to put the line-numbers; possible values are (none, left, right)
	numbersep=5pt,                   % how far the line-numbers are from the code
	numberstyle=\tiny\color{mygray}, % the style that is used for the line-numbers
	rulecolor=\color{black},         % if not set, the frame-color may be changed on line-breaks within not-black text (e.g. comments (green here))
	showspaces=false,                % show spaces everywhere adding particular underscores; it overrides 'showstringspaces'
	showstringspaces=false,          % underline spaces within strings only
	showtabs=false,                  % show tabs within strings adding particular underscores
	stepnumber=1,                    % the step between two line-numbers. If it's 1, each line will be numbered
	stringstyle=\color{mymauve},     % string literal style
	tabsize=2,	                   % sets default tabsize to 2 spaces
	title=\lstname                   % show the filename of files included with \lstinputlisting; also try caption instead of title
}

\lstdefinestyle{customc}{
	belowcaptionskip=1\baselineskip,
	breaklines=true,
	frame=L,
	xleftmargin=\parindent,
	language=C,
	showstringspaces=false,
	basicstyle=\footnotesize\ttfamily,
	keywordstyle=\bfseries\color{green!40!black},
	commentstyle=\itshape\color{purple!40!black},
	identifierstyle=\color{blue},
	stringstyle=\color{orange},
}

\lstdefinestyle{customasm}{
	belowcaptionskip=1\baselineskip,
	frame=L,
	xleftmargin=\parindent,
	language=[x86masm]Assembler,
	basicstyle=\footnotesize\ttfamily,
	commentstyle=\itshape\color{purple!40!black},
}

\lstset{escapechar=@,style=customc}

\begin{document}

\section*{}

\textbf{8.3} The (positive) semicharacteristic function of a set $A$ is the function $c$ such
that $c(a) = 1$ if $a$ is in $A$, and $c(a)$ is undefined otherwise. Show that a set $A$ is
recursively enumerable if and only if its semicharacteristic function is recursive.

Solution:

If $c$ is recursive, and $A$ is the domain of recursive function $c$, then $A$ is semirecursive.

\section*{}

\textbf{8.10} Give an example of a recursive partial function f such that f
cannot be extended to a recursive total function, or in other words, such that there is no
recursive total function g such that g(x) = f (x) for all x in the domain of f.

Let set $D_f$ denotes domain of partial function $f$. If $D_f$ is recursive
then $g$ can extend $f$:
$$
g(x) = \begin{cases}
	f(x) & x \in D_f \\
	0 & x \not\in D_f
\end{cases}
$$

If $D_f$ is not recursive, then $g$ cannot be recursive. Since $f$ is recursive,
obviously $D_f$ is semirecursive, but it still can be non-recursive. Construct
$$
f(x) = 1, x \in D_f
$$
where $D_f(x) \leftrightarrow F(x, x) = 0$, $F$ is universal function. $F(x, x) = 0$ is not a
recursive relation, thus $g(x)$ cannot be recursive.

\section*{}

\textbf{9.5} Prove that every formula $F$ has a formation sequence in which the only formulas
that appear are subformulas of $F$, and the number of formulas that appear is no greater than
the number of symbols in $F$.

Solution:

If $F$ is atomic, obviously $F$ is the only formula in its formation sequence,
and it do be its own subformula. The number of formula $1$ is not greater than
the number of symbols in $F$.

If $F$ is not atomic:
\begin{enumerate}
	\item Base step
	\begin{itemize}
		\item $F = \lnot G$, $G$ and its subformulas are subformulas of $F$.
		\item $F = (G \land H)$, $G$, $H$ and their subformulas are subformulas of $F$.
		\item $F = (G \lor H)$, $G$, $H$ and their subformulas are subformulas of $F$.
		\item $F = \forall G$, $G$ and its subformulas are subformulas of $F$.
		\item $F = \exists G$, $G$ and its subformulas are subformulas of $F$.
	\end{itemize}
	
	\item Induction step
\end{enumerate}


A formula $G$ in sequence is a subformula of later formula $\lnot G$, $(G \land H)$, $(G \lor H)$,
$\exists x\; G$, or $\forall x\; G$.

If $F$ is not atomic, then it is obtained by some earlier formulas by
negation, conjunction, disjunction, or universal or existential quantification,
which are subformulas of $F$ themselves.  
Every non-atomic formula $G$ are obtained by connect or quantify earlier
formulas, and these earlier formulas and subformulas of these earlier formulas
are all subformulas of $G$. Because $F$ is the last formula is formation
sequence, every earlier formula are subformula of it.

Assuming there is only one symbol in every atomic formula. Every step (negation,
conjunction, disjunction, universal, existential) we apply to a formula, we
added at least one symbol, not to mention there could be more symbols in atomic
formula. Thus, the number of formulas that appear is no greater than
the number of symbols in $F$.

\end{document}
