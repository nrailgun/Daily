\documentclass{ctexart}

\usepackage{amsmath}
\usepackage{amssymb}
\usepackage{amsfonts}
\usepackage{mathabx}
\usepackage{listings}
\usepackage{bussproofs}
\usepackage{tikz}
\usetikzlibrary{automata,positioning}

\usepackage{color}

\definecolor{mygreen}{rgb}{0,0.6,0}
\definecolor{mygray}{rgb}{0.5,0.5,0.5}
\definecolor{mymauve}{rgb}{0.58,0,0.82}

\lstset{ %
	backgroundcolor=\color{white},   % choose the background color; you must add \usepackage{color} or \usepackage{xcolor}
	basicstyle=\footnotesize,        % the size of the fonts that are used for the code
	breakatwhitespace=false,         % sets if automatic breaks should only happen at whitespace
	breaklines=true,                 % sets automatic line breaking
	captionpos=b,                    % sets the caption-position to bottom
	commentstyle=\color{mygreen},    % comment style
	deletekeywords={...},            % if you want to delete keywords from the given language
	escapeinside={\%*}{*)},          % if you want to add LaTeX within your code
	extendedchars=true,              % lets you use non-ASCII characters; for 8-bits encodings only, does not work with UTF-8
	frame=single,	                   % adds a frame around the code
	keepspaces=true,                 % keeps spaces in text, useful for keeping indentation of code (possibly needs columns=flexible)
	keywordstyle=\color{blue},       % keyword style
	language=Octave,                 % the language of the code
	otherkeywords={*,...},           % if you want to add more keywords to the set
	numbers=left,                    % where to put the line-numbers; possible values are (none, left, right)
	numbersep=5pt,                   % how far the line-numbers are from the code
	numberstyle=\tiny\color{mygray}, % the style that is used for the line-numbers
	rulecolor=\color{black},         % if not set, the frame-color may be changed on line-breaks within not-black text (e.g. comments (green here))
	showspaces=false,                % show spaces everywhere adding particular underscores; it overrides 'showstringspaces'
	showstringspaces=false,          % underline spaces within strings only
	showtabs=false,                  % show tabs within strings adding particular underscores
	stepnumber=1,                    % the step between two line-numbers. If it's 1, each line will be numbered
	stringstyle=\color{mymauve},     % string literal style
	tabsize=2,	                   % sets default tabsize to 2 spaces
	title=\lstname                   % show the filename of files included with \lstinputlisting; also try caption instead of title
}

\lstdefinestyle{customc}{
	belowcaptionskip=1\baselineskip,
	breaklines=true,
	frame=L,
	xleftmargin=\parindent,
	language=C,
	showstringspaces=false,
	basicstyle=\footnotesize\ttfamily,
	keywordstyle=\bfseries\color{green!40!black},
	commentstyle=\itshape\color{purple!40!black},
	identifierstyle=\color{blue},
	stringstyle=\color{orange},
}

\lstdefinestyle{customasm}{
	belowcaptionskip=1\baselineskip,
	frame=L,
	xleftmargin=\parindent,
	language=[x86masm]Assembler,
	basicstyle=\footnotesize\ttfamily,
	commentstyle=\itshape\color{purple!40!black},
}

\lstset{escapechar=@,style=customc}

\begin{document}

\section*{}
\textbf{12.4} Give an example of a sentence whose spectrum is the set of all positive integers
divisible by three.

\subsubsection*{Solution:}
Example:
$$
\begin{array}{rl}
& \forall p\; ( \exists b\; B(p, b) \land \exists h\; H(p, h) ) \\
\land & \forall b\; \exists p\; B(p, b) \\
\land & \forall h\; \exists p\; H(p, h) \\
\land & \forall p\; \forall b_1\; \forall b_2\; ( B(p, b_1) \land B(p, b_2) \to b_1 = b_2 ) \\
\land & \forall p\; \forall h_1\; \forall h_2\; ( H(p, h_1) \land H(p, h_2) \to h_1 = h_2 ) \\
\land & \forall b\; \forall p_1\; \forall p_2\; ( B(p_1, b) \land B(p_2, b) \to p_1 = p_2 ) \\
\land & \forall h\; \forall p_1\; \forall p_2\; ( H(p_1, h) \land H(p_2, h) \to p_1 = p_2 )
\end{array}
$$

Intuitively, we let $p$ denote a person, $b$ denote one's brain, and $h$ denote one's heart. Then,
everybody gets one and only one brain, one and only one heart, and every heart and brain is for someone.
Once you add a element to the domain, you have to add the other two.

\section*{}
\textbf{12.7} Let L be a language whose only nonlogical symbols are a two-place function
symbol § and a two-place predicate <. Let P be the interpretation of this
language in which the domain is the set of positive real numbers, the denotation
of § is the usual multiplication operation, and the denotation of < is the usual
order relation. Let Q be the interpretation of this language in which the domain
is the set of all real numbers, the denotation of § is the usual addition operation,
and the denotation of < is the usual order relation. Show that P and Q are
isomorphic.

\subsubsection*{Solution:}

For every constant $c$,
$$
j(c^{\cal P}) = \log c^{\cal P} = c^{\cal Q}
$$
, where $j$ is bijection.

For $<$ and positive real number $x$, $y$ in $|\cal P|$,
$$
x <^{\cal P} y \iff \log x <^{\cal Q} \log y
$$

For $\S$ and postive real number $x$, $y$ in $|\cal P|$,
$$
j(\S^{\cal Q}(x, y)) = j(x + y) =  \log(x + y) =  \log x \times \log y =  \S^{\cal P}(j(x), j(y))
$$
Thus, $\cal P$ and $\cal Q$ are isomorphic.

\section*{}
\textbf{12.10} Consider the language with just the one nonlogical symbol ≡ and the sentence
Eq whose models are precisely the sets with equivalence relations, as in the
examples in section 12.2.
\begin{itemize}
\item[(a)] For each n, indicate how to write down a sentence B n such that the
models of Eq \& B n will be sets with equivalence relations having at least
n equivalence classes
\item[(b)] For each n, indicate how to write down a formula F n (x) such that in a
model of Eq, an element a of the domain will satisfy F n (x) if and only if
there are at least n elements in the equivalence class of a.
\item[(c)] For each n, indicate how to write down a sentence C n that is true in a
model of Eq if and only if there are exactly n equivalence classes.
\item[(d)] For each n, indicate how to write down a formula G n (x) that is satisfied
by an element of the domain if and only if its equivalence class has
exactly n elements.
\end{itemize}

\subsubsection*{Solution:}
\begin{itemize}

\item[(a)]
Define $B_n$ as
\begin{gather*}
B_n = \forall x_1\; \forall x_2\; \dots, \forall x_{n-1}\; \exists x_n\;
\lnot(x_1 \equiv x_n) \land \dots \land \lnot(x_{n-1} \equiv x_n)
\end{gather*}

\item[(b)]
Define $F_n(x)$ as
\begin{gather*}
F_n(x) = \exists x_1\; \exists x_2\; \dots \exists x_{n-1}\; \\
\lnot (x_1 \equiv x_2) \land \dots \land \lnot (x_1 \equiv x_{n-1}) \land \dots \land \lnot (x_{n-2} \equiv x_{n-1}) \\
\land (x \equiv x_1) \land (x \equiv x_2) \land \dots \land (x \equiv x_{n-1})
\end{gather*}

\item[(c)]
Define $C_n$ as
$$
C_n = B_n \land (\lnot B_{n+1})
$$

\item[(d)]
Define $G_n(x)$ as
$$
G_n(x) = F_n(x) \land (\lnot F_{n+1} x)
$$

\end{itemize}

\section*{}
\textbf{14.8}
Show that $\forall x (Fx \& Gx)$ is deducible from $\forall xFx \land \forall xGx$

\subsubsection*{Solution:}

\begin{prooftree}
	\AxiomC{*}
	\UnaryInfC{$Fx_1, \forall x\; Gx \Rightarrow Fx_1$}
	\UnaryInfC{$\forall x\; Fx, \forall x\; Gx \Rightarrow Fx_1$}

	\AxiomC{*}
	\UnaryInfC{$\forall x\; Fx, Gx_1 \Rightarrow Gx_1$}
	\UnaryInfC{$\forall x\; Fx, \forall x\; Gx \Rightarrow Gx_1$}
	
	\BinaryInfC{$\forall x\; Fx, \forall x\; Gx \Rightarrow Fx_1 \land Gx_1$}
	\UnaryInfC{$\forall x\; Fx, \forall x\; Gx \Rightarrow \forall x\; (Fx \land Gx)$}
	\UnaryInfC{$\forall x\; Fx \land \forall x\; Gx \Rightarrow \forall x\; (Fx \land Gx)$}
\end{prooftree}

\section*{}
\textbf{14.9}
Show that the transitivity of identity, $\forall x\; \forall y\; \forall z\; (x = y \land y = z \to x = z)$ is
demonstrable.

\subsubsection*{Solution:}
\begin{prooftree}
\AxiomC{*}
\UnaryInfC{$x = z \Rightarrow x = z$}
\UnaryInfC{$x = y, y = z \Rightarrow x = z$}
\UnaryInfC{$x = y \land y = z \Rightarrow x = z$}
\UnaryInfC{$\emptyset \Rightarrow \lnot (x = y \land y = z), x = z$}
\UnaryInfC{$\emptyset \Rightarrow \lnot (x = y \land y = z) \lor x = z$}
\UnaryInfC{$\emptyset \Rightarrow x = y \land y = z \to x = z$}
\UnaryInfC{$\emptyset \Rightarrow \forall z\; (x = y \land y = z \to x = z)$}
\UnaryInfC{$\emptyset \Rightarrow \forall y\; \forall z\; (x = y \land y = z \to x = z)$}
\UnaryInfC{$\emptyset \Rightarrow \forall x\; \forall y\; \forall z\; (x = y \land y = z \to x = z)$}
\end{prooftree}

\section*{}
\textbf{14.13}
Consider adding one or the other of the following rules to (R0)–(R8):
Show that a sequent is derivable on adding (R11) if and only if it is derivable
on adding (R12).

\subsubsection*{Solution:}

\begin{prooftree}
\AxiomC{$\Gamma \Rightarrow \{ A \} \cup \Delta$}
\UnaryInfC{$\Gamma \cup \{ \lnot A \} \Rightarrow \Delta$}
\AxiomC{$\Gamma \cup \{ A \} \Rightarrow \Delta$}
\BinaryInfC{$\Gamma \cup \{ (A \lor \lnot A)\} \Rightarrow \Delta$}
\end{prooftree}
Thus, if adding (R11) makes sequence derivable, so does adding (R12).
	
\end{document}
